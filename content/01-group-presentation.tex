\chapter{Introduction}

This chapter is based on the first chapter of \cite{magnusCombinatorialGroupTheory2004}. This chapter will be an introduction of what groups are and how they are generated.\\

We recall in group theory that a group $(G,\cdot)$ is a non-empty set $G$ of elements with a binary operation $\cdot$ for which the next axioms are satisfied:

\begin{itemize}
  \item \textbf{Closure:} For all $a,b \in G$, $c$ such that $a\cdot b = c$ implies that $c \in G$.

  \item \textbf{Associativity:} The operation $\cdot$ is associative, which means that for any elements $a,b,c \in G$:
    $$(ab)c = a(bc)$$

  \item \textbf{Identity element:} There exists an element of $G$ noted $1$ for which:

  $$a\cdot 1 = 1\cdot a = a$$

  \item \textbf{Inverse element:} For any $a \in G$ there exists an \textit{element} $a^{-1}$ for which:
  $$ a\cdot a^{-1} = a^{-1} \cdot a = 1 $$
\end{itemize}

We know two ways of defining a group; defining a \textit{symmetry} of a set and if it is presented by generators and relators.

\section{Symmetric groups}


\definition{The \textit{symmetric group} on the set $G$ is the group whose elements are permutations of the elements of $G$ and its operation is the permutation composition. If $G = \{1,\dots, n\}$ we call it $S_n$.}

\section{Presentation of groups}

A group
