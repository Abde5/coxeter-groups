%%%%%%%%%%%%%%%%%%%% book.tex %%%%%%%%%%%%%%%%%%%%%%%%%%%%%
%
% sample root file for the chapters of your "monograph"
%
% Use this file as a template for your own input.
%
%%%%%%%%%%%%%%%% Springer-Verlag %%%%%%%%%%%%%%%%%%%%%%%%%%


% RECOMMENDED %%%%%%%%%%%%%%%%%%%%%%%%%%%%%%%%%%%%%%%%%%%%%%%%%%%
\documentclass[envcountsame,envcountchap]{svmono}
%\documentclass[envcountsame,envcountchap]{svmono}

% choose options for [] as required from the list
% in the Reference Guide, Sect. 2.2

\usepackage{makeidx}         % allows index generation
\usepackage{graphicx}        % standard LaTeX graphics tool
\usepackage{amsmath,amssymb}         % matrices
\usepackage{enumerate}
                             % when including figure files
\usepackage{multicol}        % used for the two-column index
\usepackage[bottom]{footmisc}% places footnotes at page bottom
% etc.
% see the list of further useful packages
% in the Reference Guide, Sects. 2.3, 3.1-3.3


\DeclareMathOperator{\End}{End}
\DeclareMathOperator{\Aut}{Aut}
\DeclareMathOperator{\Hom}{Hom}
\DeclareMathOperator{\support}{supp}


% NEW COMMANDS

%It is standard in Latex to write "macros" which are shorthand for an entire series of instructions. Here are some examples

%Number sets
\newcommand{\N}{\mathbb N}
%So typing \N produces the correct mathematical symbol for the natural numbers
\newcommand{\Z}{\mathbb Z}
\newcommand{\Q}{\mathbb Q}
\newcommand{\R}{\mathbb R}
\newcommand{\C}{\mathbb C}
\newcommand{\K}{\mathbb K}
%notations quelquonques
\newcommand{\tg}[1]{\textbf{#1}}
\newcommand{\ub}[1]{\overline{#1}}

%notations des objets simples
\newcommand{\es}{\emptyset}
\newcommand{\nes}{$\not= \emptyset$}
\newcommand{\sub}{\subset}
\newcommand{\norm}[2]{\lVert #1 \lVert_{#2}}
\newcommand{\vect}[2]{(#1_1,#1_2, \dots, #1_#2)}
\newcommand{\modu}[1]{\lvert#1\lvert}
\newcommand{\B}[3]{B_{#1}\big(#2,#3\big[}
%notations mathématiques
\newcommand{\lb}{\lbrack}
\newcommand{\rb}{\rbrack}
\newcommand{\lv}{\lVert}
%limits and sum
\newcommand{\s}[2]{\sum\limits_{#1}^{#2}}
\newcommand{\li}[2]{\xrightarrow[#1\rightarrow#2]{}}
\newcommand{\lis}[1]{\xrightarrow[n\rightarrow+\infty]{#1}}
\newcommand{\lif}[1]{\xrightharpoonup[n\rightarrow+\infty]{#1}}
\newcommand{\lic}[3]{\xrightarrow[#1\rightarrow#2]{#3}}

\newcommand{\bcup}[2]{\bigcup\limits_{#1}^{#2}}
\newcommand{\bcap}[2]{\bigcap\limits_{#1}^{#2}}

\newcommand{\inv}[1]{\frac{1}{#1}}
\newcommand{\prods}[2]{\langle\qq #1\qq,\qq#2\qq\rangle}

\newcommand{\restr}[2]{#1_{\mkern 2mu \vrule height 2ex\mkern2mu #2} }
\newcommand{\quot}[2]{{\raisebox{.2em}{$#1$}\left/\raisebox{-.2em}{$#2$}\right.}}
\newcommand{\limite}[2]{\underset{#1\rightarrow#2}{\text{lim}}}
\newcommand{\espp}[2]{Ker\big(u-{#1} Id_{#2}\big)}
\newcommand{\fct}[4]{\qq:\qq #1\qq\longrightarrow\qq #2\qq :\qq #3\qq \mapsto\qq #4}

\newcommand{\lam}{\lambda}
\newcommand{\q}{\quad}
\newcommand{\qq}{\text{ }}

\newcommand{\liste}[2]{#1_1, #1_2,..,#1_{#2}}

\newcommand{\maxx}[1]{\underset{#1}{\text{max}}}
\newcommand{\minn}[1]{\underset{#1}{\text{min}}}
\newcommand{\supp}[1]{\underset{#1}{\text{sup}}}
\newcommand{\inff}[1]{\underset{#1}{\text{inf}}}

\newcommand{\fctt}[2]{\qq:\qq#1\qq\rightarrow\qq#2}
\newcommand{\liminff}[1]{\underset{#1\rightarrow+\infty}{\text{liminf}}}
\newcommand{\limsupp}[1]{\underset{#1\rightarrow+\infty}{\text{limsup}}}

\newcommand{\adh}[2]{\text{Adh}_{#1}\big(#2\big)}
\newcommand{\wed}[3]{#1_#2\wedge\dots \wedge #1_#3}


\makeindex             % used for the subject index
                       % please use the style svind.ist with
                       % your makeindex program


%%%%%%%%%%%%%%%%%%%%%%%%%%%%%%%%%%%%%%%%%%%%%%%%%%%%%%%%%%%%%%%%%%%%%

\begin{document}

\author{MATH-F-427 students}
\title{Coxeter groups}
\subtitle{Course notes}
\maketitle

\frontmatter%%%%%%%%%%%%%%%%%%%%%%%%%%%%%%%%%%%%%%%%%%%%%%%%%%%%%%

\tableofcontents

The following developments aim to prove the theorem \ref{thm central}. 

\begin{definition}
Let $V = \bigoplus_{n \ge 0} V_n$ a graded vector space. The Hilbert series of $V$ is defined as
\begin{equation}
\text{Hilb}_V (t) = \sum_{n\ge 0} (\dim V_n)t^n \in \mathbb{Q}[t]
\end{equation}
\end{definition}

\begin{example}
Consider $\mathbb{C}[\theta_1, \ldots , \theta_n ] \subset \mathbb{C}[\bar{x}]$. We have
\begin{equation}
\begin{split}
\text{Hilb}_{\mathbb{C}[\theta_1, \ldots , \theta_n ]}(t) &= \frac{1}{(1-t^{d^1})(1-t^{d_2}) \ldots (1-t^{d^n})} \\
&= (1+ t^{d_1} + t^{2d_1} + t^{3d_1} + \ldots ) (1+ t^{d_2} + t^{2d_2} + \ldots ) \ldots \\
&= \sum_{d\ge 0} \left( \sum_{(\alpha_1, \ldots, \alpha_n) \in \mathbb{N}^n: \alpha_1 d_1 + \ldots + \alpha_n d_n = d} \underbrace{t^{\alpha_1 d_1} t^{\alpha_2 d_2} \ldots t^{\alpha_n d_n}}_{t^d} \right) \\
\end{split}
\end{equation} Furthermore,
\begin{equation}
\mathbb{C}[\theta_1, \ldots , \theta_n ]_d = \text{Span}_\mathbb{C} \{ \theta_1^{\alpha_1} \theta_2^{\alpha_2} \ldots \theta_n^{\alpha_n} | \alpha_1 d_1 + \ldots + \alpha_n d_n = d \}
\end{equation} and 
\begin{equation}
\dim \mathbb{C}[\theta_1, \ldots , \theta_n ]_d = \sharp \{ (\alpha_1, \ldots, \alpha_n) \in \mathbb{N}^n | \alpha_1 d_1 + \ldots + \alpha_n d_n = d\}
\end{equation}
\end{example}

\begin{theorem}\label{Molien} [Molien]
Let $G \subset GL(\mathbb{C}^n)$ be a finite group. We have
\begin{equation}
\text{Hilb}_{\mathbb{C}[\bar{x}]^G} (t) = \frac{1}{|G|} \sum_{\pi \in G} \frac{1}{\det (I- \pi t )}
\end{equation}
\end{theorem}
\begin{proof}
Consider $(\mathbb{C}^n)^G = \{v \in \mathbb{C}^n | \pi v = v , \forall \pi \in G \}$. We define 
\begin{equation}
P_G := \frac{1}{|G|} \sum_{\pi \in G} \pi \in \text{End}(\mathbb{C}^n)
\end{equation} This operator is the projection of $\mathbb{C}^n$ onto $(\mathbb{C}^n)^G$. We have $P_G^2 = P_G$ and 
\begin{equation}
\begin{split}
\dim (\mathbb{C}^n)^G &= \text{rank} P_G \\
&= \text{Tr} (P_G) \\
&= \frac{1}{|G|} \sum_{\pi \in G} \text{Tr} \pi 
\end{split}
\end{equation} Recall that
\begin{equation}
\mathbb{C}[\bar{x}] = \bigoplus_{d\ge0} \mathbb{C}[\bar{x}]_d
\end{equation} For each $\pi \in G$, we write $\pi^{(d)} \in GL(\mathbb{C}[\bar{x}]_d )$ its restriction to $\mathbb{C}[\bar{x}]_d$. Not that
 \begin{equation}
 \mathbb{C}[\bar{x}]^G = \bigoplus_{d\ge0} \mathbb{C}[\bar{x}]_d^G
 \end{equation} Now, we can identify $\mathbb{C}^n$ with $\mathbb{C}[\bar{x}]_1$. Let $\pi \in G$ and $\ell_{\pi, 1}, \ldots , \ell_{\pi, n} \in \mathbb{C}[\bar{x}]_1$ a basis of eigenvectors associated with eigenvalues $\lambda_{\pi, 1}, \ldots , \lambda_{\pi, n}$, respectively. So $\{ \ell_{\pi,1}^{d_1}, \ldots , \ell_{\pi,n}^{d_n} |d_1 + \ldots + d_n = d \}$ is a basis of $\mathbb{C}[\bar{x}]_d$. Here, $ \ell_{\pi,1}^{d_1}, \ldots , \ell_{\pi,n}^{d_n}$ are eigenvectors of $\pi^{(d)}$ associated with eigenvalues $\lambda_{\pi, 1}^{d_1}, \ldots ,\lambda_{\pi, n}^{d_n}$. Therefore,
 \begin{equation}
 \begin{split}
 \dim \mathbb{C}_d^G &= \frac{1}{|G|} \sum_{\pi \in G} \text{Tr} \pi^{(d)} \\
 &= \frac{1}{|G|} \sum_{\pi \in G} \left( \sum_{(d_1, \ldots , d_n) \in\mathbb{N}^n: d_1 + \ldots + d_n = d} \lambda_{\pi, 1}^{d_1} \ldots \lambda_{\pi, n}^{d_n} \right)
 \end{split}
 \end{equation} Hence,
 \begin{equation}
 \begin{split}
 \text{Hilb}_{\mathbb{C}[\bar{x}]^d}(t) &= \sum_{d\ge 0} t^d \frac{1}{|G|} \sum_{\pi \in G}  \left( \sum_{(d_1, \ldots , d_n) \in\mathbb{N}^n: d_1 + \ldots + d_n = d} \lambda_{\pi, 1}^{d_1} \ldots \lambda_{\pi, n}^{d_n} \right) \\
 &= \frac{1}{|G|} \sum_{\pi \in G}  \left( \sum_{(d_1, \ldots , d_n) \in\mathbb{N}^n} (\lambda_{\pi, 1} t)^{d_1} \ldots (\lambda_{\pi, n} t)^{d_n} \right) \\
 &=  \frac{1}{|G|} \sum_{\pi \in G} \frac{1}{(1 -\lambda_{\pi, 1} t ) \ldots ( 1- \lambda_{\pi, n} t) } \\
 &= \frac{1}{|G|} \sum_{\pi \in G} \frac{1}{\det (I - t \pi)}
 \end{split}
 \end{equation}
 \end{proof}
 
 \begin{lemma}
 Let $G \subset GL( \mathbb{C}^n)$ be a finite group. Let $r$ be be the number of reflections in $G$. Then the Laurent expansion of $\text{Hilb}_{\mathbb{C}[\bar{x}]^G}(t)$ at $t=1$ begins as 
 \begin{equation}
 \text{Hilb}_{\mathbb{C}[\bar{x}]^G}(t)= \frac{1}{|G|} (1-t)^{-n} + \frac{r}{2|G|} (1-t)^{-n+1} + \mathcal{O}((1-t)^{-n+2})
 \end{equation}
 \label{lemma number refl}
 \end{lemma}
 \begin{proof}
 By Molien's theorem \ref{Molien}, we have
 \begin{equation}
 \begin{split}
 \text{Hilb}_{\mathbb{C}[\bar{x}]^G}(t) &= \frac{1}{|G|} \sum_{\pi\in G} \frac{1}{\det (I - \pi t)} \\
 &= \frac{1}{|G|} (1-t)^{-n} + \sum_{\sigma \text{ reflections}} \frac{1}{(1-t)^{n-1}(1- \det \sigma  t)} \\
 &= \frac{1}{|G|} (1-t)^{-n} + \frac{(1-t)^{-n+1}}{|G|} \sum_{\sigma \text{ reflections}} \frac{1}{(1- \det \sigma  )} + \mathcal{O}((1-t)^{-n+2})
 \end{split}
\end{equation}  Furthermore, 
\begin{equation}
\begin{split}
2 \sum_{\sigma \text{ reflections}} \frac{1}{(1 - \det \sigma)} &= \sum_{\sigma \text{ reflections}} \left( \frac{1}{(1 - \det \sigma)} +  \frac{1}{(1 - \det \sigma^{-1})} \right) \\
&= \sum_{\sigma \text{ reflections}} \frac{1 - \det \sigma^{-1} + 1 - \det \sigma}{(1 - \det \sigma)(1 - \det \sigma^{-1})} \\
&=\sum_{\sigma \text{ reflections}} 1 \\
&= r
\end{split}
\end{equation} This concludes the proof of the lemma.
 \end{proof}
 
\begin{corollary}
Let $G \subset GL(\mathbb{C}^n)$ be a finite group and $\mathbb{C}[\bar{x}]^G = \mathbb{C}[\theta_1, \ldots , \theta_n ]$ with $\theta_i$ algebraically independent and homogeneous of degrees $d_i$. Then,
\begin{equation}
|G| = d_1 d_2 \ldots d_n \quad \text{and} \quad \sum_{i=1}^n (d_i- 1) = r = \sharp\text{ of reflections}
\end{equation}
\label{corallary useful}
\end{corollary}
\begin{proof}
We have
\begin{equation}
\begin{split}
\text{Hilb}_{\mathbb{C}[\theta_1, \ldots, \theta_n]}(t) &= \frac{1}{(1-t^{d_1}) \ldots (1- t^{d_n})} \\
&= \frac{1}{(1-t)^n} \frac{1}{(1+ t + \ldots + t^{d_1 -1} ) \ldots (1+ t + \ldots + t^{d_n -1} )} \\
&= \frac{1}{(1-t)^n} \left( \frac{1}{d_1 \ldots d_n} + \frac{\begin{pmatrix}
d_1 \\
2          
\end{pmatrix}}{d_1^2 d_2 \ldots d_n} + \ldots + \frac{\begin{pmatrix}
d_n \\
n          
\end{pmatrix}}{d_1 \ldots d_{n-1} d_n^2} \right) (1-t) +\mathcal{O}( (1-t)^{-n+2} ) \\
&= \frac{1}{d_1 \ldots d_n} (1-t)^{-n} + \frac{\sum_{d=1}^n (d_i - 1)}{2 d_1 \ldots d_n} (1-t)^{-n+1} + \mathcal{O}((1-t)^{-n+2})
\end{split}
\end{equation} From lemma \ref{lemma number refl}, we obtain the results.
\end{proof}
 
\begin{lemma}
If $f_1, \ldots, f_n \in \mathbb{C}[\bar{x}]$ are algebraically independent over $G$, then $\det \left( \frac{\partial f_i}{\partial x_j} \right)_{1 \le i,j \le n} \neq 0$. 
\label{useful lemma}
\end{lemma}
\begin{proof}
We know that $\mathbb{C}[x_1, \ldots , x_n]$ has transcendent degree $n$. Hence, $x_i, f_1, \ldots, f_n$ are algebraically dependent. Let $h_i (y_0, y_1, \ldots, y_n)$ be a polynomial of maximal degree such that $h_i (x_i, f_1, \ldots , f_n ) = 0$. For $k \in \{1, 2, \ldots, n \}$, we have
\begin{equation}
\frac{\partial h_i (x_i, f_1, \ldots, f_n)}{\partial x_k} = \sum_{j=1}^n \frac{\partial h_i }{\partial y_j}(x_i, f_1, \ldots, f_n) \frac{\partial f_j}{\partial x_k} + \delta_{ik} \frac{\partial h_i}{\partial y_0} (x_i, f_1, \ldots, f_n) = 0
\label{equation proof}
\end{equation} Since the $f_i$ are algebraically independent, $h_i$ has positive degree in $y_0$. Hence, 
\begin{equation}
\frac{\partial h_i (y_0, y_1, \ldots, y_n)}{\partial y_0}  \neq 0
\end{equation} Since it has smaller degree, we have
\begin{equation}
\frac{\partial h_i (x_i, f_1, \ldots, f_n)}{\partial y_0}  \neq 0
\end{equation} From \eqref{equation proof}, we find
\begin{equation}
\frac{\partial h_i }{\partial y_j}(x_i, f_1, \ldots, f_n) \frac{\partial f_j}{\partial x_k} = - \delta_{ik} \frac{\partial h_i}{\partial y_0} (x_i, f_1, \ldots, f_n) 
\end{equation} Since
\begin{equation}
\det \left( - \delta_{ik} \frac{\partial h_i}{\partial y_0} (x_i, f_1, \ldots, f_n)  \right) \neq 0
\end{equation} we find the desired result
\begin{equation}
\det \left( \frac{\partial f_j}{\partial x_k} \right) \neq 0
\end{equation}
\end{proof}



\begin{theorem}
Let $G \subset GL(\mathbb{C}^n )$ be a finite group and $\mathbb{C}^G = \mathbb{C}[\theta_1, \ldots , \theta_n ]$ where $\theta_i$ are algebraically independent homogeneous of degrees $d_i$. Then $G$ is a reflection group. 
\label{thm central}
\end{theorem}
\begin{proof}
Let $H$ be the subgroup generated by the reflections of $G$. Using previous result, we know that $\mathbb{C}[\bar{x}]^H = \mathbb{C}[\Psi_1, \ldots , \Psi_n ]$, with $\Psi_i$ algebraically independent homogeneous of degree $e_i$. Clearly,
\begin{equation}
\mathbb{C}[\bar{x}]^G \subset \mathbb{C}[\bar{x}]^H = \mathbb{C}[\Psi_1, \ldots, \Psi_n]
\end{equation} Hence, $\theta_i$ is a polynomial in $\Psi_1, \ldots, \Psi_n$ for all $i$. By lemma \ref{useful lemma}, we have
\begin{equation}
\det \left( \frac{\partial \theta_i}{\partial \Psi_j}\right) \neq 0
\end{equation} Thus, for some permutation $\pi$, we have 
\begin{equation}
\frac{\partial \theta_{\pi(1)}}{\partial \Psi_1} \frac{\partial \theta_{\pi(2)}}{\partial \Psi_2} \ldots \frac{\partial \theta_{\pi(n)}}{\partial \Psi_n} \neq 0
\end{equation} This means that $\Psi_i$ actually occurs in $\theta_{\pi(i)}$. Hence, 
\begin{equation}
e_i = \deg \Psi_i \le \det \theta_{\pi(i)} = d_{\pi(i)}
\end{equation} Let $r$ be the number of reflections in $G$, and so also in $H$. By corollary \ref{corallary useful}, we obtain
\begin{equation}
r = \sum_{i=1}^n (d_i - 1) =  \sum_{i=1}^n (e_i - 1) 
\end{equation} Therefore, we must have $e_i = d_{\pi(i)}$ for all $i$. Using corollary \ref{corallary useful} again, we have
\begin{equation}
|G| = d_1 \ldots d_n = e_1 \ldots e_n = |H|
\end{equation} We deduce $H = G$. 



\end{proof}



\mainmatter%%%%%%%%%%%%%%%%%%%%%%%%%%%%%%%%%%%%%%%%%%%%%%%%%%%%%%%


	
\end{document}	